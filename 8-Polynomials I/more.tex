\documentclass[12pt,a4paper,twoside]{article}
\usepackage{amsmath}
\usepackage{amssymb}
\usepackage{amsthm}
\usepackage{array}
\usepackage{amsfonts}
\usepackage{ctable,booktabs}
%\usepackage[pdftex]{graphicx}
\usepackage{enumerate}
\usepackage{fancyhdr}
\usepackage{float}
\usepackage{fullpage}
\usepackage[margin=1in]{geometry}
\usepackage{mathrsfs}%Math script
\usepackage{sidecap}
\usepackage{tabularx} 
\usepackage{verbatim}
\usepackage{wrapfig}
\usepackage[all,cmtip]{xy}%Commutative diagrams
	\headheight 0cm
	\setlength{\headsep}{18pt}
	\setlength{\headheight}{15.2pt}

\newtheoremstyle{norm}
{3pt}
{3pt}
{}
{}
{\bf}
{:}
{.5em}
{}

\theoremstyle{norm}
\newtheorem{thm}{Theorem}[section]
\newtheorem{lem}[thm]{Lemma}
\newtheorem{df}[thm]{Definition}
\newtheorem{rem}[thm]{Remark}
\newtheorem{st}{Step}
\newtheorem{pr}[thm]{Proposition}
\newtheorem{cor}[thm]{Corollary}
\newtheorem{conj}[thm]{Conjecture}
\newtheorem{clm}[thm]{Claim}
\newtheorem{exr}[thm]{Exercise}
\newtheorem{ex}[thm]{Example}
\newtheorem{prb}[thm]{Problem}

%Math blackboard, fraktur, and script commonly used letters
\newcommand{\A}[0]{\mathbb{A}}
\newcommand{\C}[0]{\mathbb{C}}
\newcommand{\sC}[0]{\mathcal{C}}
\newcommand{\cE}[0]{\mathscr{E}}
\newcommand{\F}[0]{\mathbb{F}}
\newcommand{\cF}[0]{\mathscr{F}}
\newcommand{\sF}[0]{\mathscr{F}}
\newcommand{\cG}[0]{\mathscr{G}}
\newcommand{\sH}[0]{\mathscr H}
\newcommand{\Hq}[0]{\mathbb{H}}
\newcommand{\N}[0]{\mathbb{N}}
\newcommand{\Pj}[0]{\mathbb{P}}
\newcommand{\sO}[0]{\mathcal{O}}
\newcommand{\cO}[0]{\mathscr{O}}
\newcommand{\Q}[0]{\mathbb{Q}}
\newcommand{\R}[0]{\mathbb{R}}
\newcommand{\Z}[0]{\mathbb{Z}}
%Lowercase
\newcommand{\ma}[0]{\mathfrak{a}}%ideal a
\newcommand{\mb}[0]{\mathfrak{b}}
\newcommand{\fg}[0]{\mathfrak{g}}
\newcommand{\vi}[0]{\mathbf{i}}%vector i
\newcommand{\vj}[0]{\mathbf{j}}
\newcommand{\vk}[0]{\mathbf{k}}
\newcommand{\mm}[0]{\mathfrak{m}}%ideal m
\newcommand{\mfp}[0]{\mathfrak{p}}
\newcommand{\mq}[0]{\mathfrak{q}}
\newcommand{\mr}[0]{\mathfrak{r}}
%More sequences of letters
\newcommand{\fq}[0]{\mathbb{F}_q}
\newcommand{\fqt}[0]{\mathbb{F}_q^{\times}}
\newcommand{\sll}[0]{\mathfrak{sl}}
%Shortcuts for symbols
\newcommand{\nin}[0]{\not\in}
\newcommand{\opl}[0]{\oplus}
\newcommand{\ot}[0]{\otimes}
\newcommand{\rc}[1]{\frac{1}{#1}}
\newcommand{\sub}[0]{\subset}
\newcommand{\subeq}[0]{\subseteq}
\newcommand{\supeq}[0]{\supseteq}
\newcommand{\nsubeq}[0]{\not\subseteq}
\newcommand{\nsupeq}[0]{\not\supseteq}
%Arrows
\newcommand{\lar}[0]{\leftarrow}
\newcommand{\ra}[0]{\rightarrow}
\newcommand{\rra}[0]{\rightrightarrow}
\newcommand{\hra}[0]{\hookrightarrow}
\newcommand{\send}[0]{\mapsto}
%Shortcuts for greek letters
\newcommand{\al}[0]{\alpha}
\newcommand{\be}[0]{\beta}
\newcommand{\ga}[0]{\gamma}
\newcommand{\Ga}[0]{\Gamma}
\newcommand{\de}[0]{\delta}
\newcommand{\ep}[0]{\varepsilon}
\newcommand{\eph}[0]{\frac{\varepsilon}{2}}
\newcommand{\ept}[0]{\frac{\varepsilon}{3}}
\newcommand{\la}[0]{\lambda}
\newcommand{\La}[0]{\Lambda}
\newcommand{\ph}[0]{\varphi}
\newcommand{\rh}[0]{\rho}
\newcommand{\te}[0]{\theta}
\newcommand{\om}[0]{\omega}
\newcommand{\Om}[0]{\Omega}
%Brackets
\newcommand{\ab}[1]{\left| {#1} \right|}
\newcommand{\an}[1]{\langle {#1}\rangle}
\newcommand{\ba}[1]{\left[ {#1} \right]}
\newcommand{\bc}[1]{\left\{ {#1} \right\}}
\newcommand{\ce}[1]{\left\lceil {#1}\right\rceil}
\newcommand{\fl}[1]{\left\lfloor {#1}\right\rfloor}
\newcommand{\pa}[1]{\left( {#1} \right)}
%Text
\newcommand{\btih}[1]{\text{ by the induction hypothesis{#1}}}
\newcommand{\bwoc}[0]{by way of contradiction}
\newcommand{\by}[1]{\text{by~(\ref{#1})}}
\newcommand{\ore}[0]{\text{ or }}
\newcommand{\wog}[0]{ without loss of generality }
\newcommand{\Wog}[0]{ Without loss of generality }
%Functions, etc.
\newcommand{\Ann}{\operatorname{Ann}}
\newcommand{\AP}{\operatorname{AP}}
\newcommand{\Ass}{\operatorname{Ass}}
\newcommand{\chr}{\operatorname{char}}
\newcommand{\cis}{\operatorname{cis}}
\newcommand{\Cl}{\operatorname{Cl}}
\newcommand{\Der}{\operatorname{Der}}
\newcommand{\End}{\operatorname{End}}
\newcommand{\Ext}{\operatorname{Ext}}
\newcommand{\Frac}{\operatorname{Frac}}
\newcommand{\FS}{\operatorname{FS}}
\newcommand{\GL}{\operatorname{GL}}
\newcommand{\Hom}{\operatorname{Hom}}
\newcommand{\chom}[0]{\mathscr{H}om}
\newcommand{\Ind}[0]{\text{Ind}}
\newcommand{\im}[0]{\text{im}}
\newcommand{\nil}[0]{\operatorname{nil}}
\newcommand{\Proj}{\operatorname{Proj}}
\newcommand{\Rad}{\operatorname{Rad}}
\newcommand{\Res}[0]{\text{Res}}
\newcommand{\sign}{\operatorname{sign}}
\newcommand{\SL}{\operatorname{SL}}
\newcommand{\Spec}{\operatorname{Spec}}
\newcommand{\Specf}[2]{\Spec\pa{\frac{k[{#1}]}{#2}}}
\newcommand{\spp}{\operatorname{sp}}
\newcommand{\spn}{\operatorname{span}}
\newcommand{\Supp}{\operatorname{Supp}}
\newcommand{\Tor}{\operatorname{Tor}}
\newcommand{\tr}[0]{\text{Tr}}
%Commutative diagram shortcuts
\newcommand{\commsq}[8]{\xymatrix{#1\ar[r]^{#6}\ar[d]^{#5} &#2\ar[d]^{#7} \\ #3 \ar[r]^{#8} & #4}}
%Makes a diagram like this
%1->2
%|    |
%3->4
%Arguments 5, 6, 7, 8 on arrows
%  6
%5  7
%  8
\newcommand{\pull}[9]{
#1\ar@/_/[ddr]_{#2} \ar@{.>}[rd]^{#3} \ar@/^/[rrd]^{#4} & &\\
& #5\ar[r]^{#6}\ar[d]^{#8} &#7\ar[d]^{#9} \\}
\newcommand{\back}[3]{& #1 \ar[r]^{#2} & #3}
%Syntax:\pull 123456789 \back ABC
%1=upper left-hand corner
%2,3,4=arrows from upper LH corner, going down, diagonal, right
%5,6,7=top row (6 on arrow)
%8,9=middle rows (on arrows)
%A,B,C=bottom row
%Other
\newcommand{\op}{^{\text{op}}}
\newcommand{\fp}[1]{^{\underline{#1}}}%Falling power
\newcommand{\rp}[1]{^{\overline{#1}}}
\newcommand{\rd}[0]{_{\text{red}}}
\newcommand{\pre}[0]{^{\text{pre}}}
\newcommand{\pf}[2]{\pa{\frac{#1}{#2}}}%Shortcut for fraction with parentheses
\newcommand{\pd}[2]{\frac{\partial #1}{\partial #2}}%Partial derivatives
%Matrices
\newcommand{\coltwo}[2]{
\left[
\begin{array} {c}
{#1}\\
{#2} 
\end{array}
\right]}
\newcommand{\matt}[4]{
\begin{pmatrix} {cc}
{#1}&{#2}\\
{#3}&{#4}
\end{pmatrix}
}
\newcommand{\smatt}[4]{
\left[\begin{smallmatrix} {cc}
{#1}&{#2}\\
{#3}&{#4}
\end{smallmatrix}\right]
}
\newcommand{\colthree}[3]{
\begin{pmatrix} {c}
{#1}\\
{#2}\\
{#3}
\end{pmatrix}
}
%Page breaks in equations
\allowdisplaybreaks[1]

\rhead{\qquad}
%\lhead{}\rhead{}\cfoot{}\chead{} 
%\lfoot[\fancyplain{}{\bfseries\thepage}]{}
%\rfoot[]{\thepage}
%\lfoot[\thepage]{}

%\setlength{\oddsidemargin}{0.55in}
%\setlength{\evensidemargin}{0.55in}


%\fi
\pagestyle{fancy}

%OMC <year>
\lhead{\textit{OMC 2011}}

% LECTURE TITLE
\chead{Polynomials}

%%%%%%%%%%
% LECTURE NUMBER
%%%%%%%%%%
\rhead{\textit{Lecture 8}}


%%%%%%%%%%
% CONTENT
%
% Here is where you place the problems and any other content. If you are making a problem set, use the \item command to create a new problem. 
%
%%%%%%%%%%
\begin{document}
\title{Lecture $8$ --- Polynomials}% !! Remember to change the lecture number
\author{Holden Lee and Josh Nichols-Barrer}
\date{1/14/11}% !! Remember to change the date
\maketitle
\thispagestyle{empty}



\section{Irreducible Polynomials}
\begin{enumerate}
\item  Let $n\geq 1$ and let $f(x)$ be a degree-$n$ integer polynomial such that for $2n-1$ distinct integer values of $x$, the value of $f(x)$ is plus or minus a prime. Show that $f(x)$ is irreducible in $\mathbb{Q}[x]$.
\item (IMO 1992?) Show that for each $n>1$ the polynomial $x^n+5x^{n-1}+3$ is irreducible in
$\mathbb Z[x]$.
\item Prove that the polynomial
\[f(x)=\frac{x^n+x^m-2}{x^{\text{gcd}(m,n)}-1}\]
is irreducible over $\mathbb Q$ for all integers $n>m>0$.
\item Let $a$ be an integer not divisible by 5.  Show that $x^5-x-a$
is irreducible in $\mathbb Z[x]$.
\item Let $m,n$ be integers with $m>0$ and $5\mid\mid n$ (in other words, $5|n$ and no larger power of 5 divides $n$). Show that $(x^4+x^2-6)^m+n$ is irreducible in $\mathbb{Q}[x]$.
\end{enumerate}

FROM JOSH
\begin{enumerate}
\item Show that the polynomial $x^{2^n}+1$ is irreducible in $\mathbb
Z[x]$.
\item Show that the polynomial $x^n-1998$ is irreducible in $\mathbb
Z[x]$.
\item Show that the polynomial $x^{p-1}+x^{p-2}+\cdots+1$ is
irreducible in $\mathbb Z[x]$.
\item Let $m$ and $n$ be positive integers.  Show that the polynomial
$x^m+y^n-z^n$ is irreducible in $\mathbb Z[x,y,z]$.
\item Let $p\equiv 3\bmod 4$ be a prime number and let $a$ and $b$ be
integers such that $p | a$ and $p || b-1$.  Show that the polynomial
$f(x) = x^{2p} + ax+b$ is irreducible in $\mathbb Z[x]$.
\item Let $a_1,a_2,\ldots,a_n$ be distinct integer numbers.  Show that
the polynomial \[(x-a_1)(x-a_2)\cdots(x-a_n)-1\] is irreducible in
$\mathbb Z[x]$.
\item Let $a_1,a_2,\ldots,a_n$ be distinct integer numbers.  Show that
the polynomial \[(x-a_1)^2(x-a_2)^2\cdots(x-a_n)^2+1\] is irreducible
in $\mathbb Z[x]$.
\end{enumerate}

\section{Blah}
\noindent\textbf{(C) Divisibility, GCD, and Irreducibility}
\begin{thm}[B\'{e}zout]
Let $K$ be a field and $f,g\in K[x]$. There exist polynomials $u,v\in K[x]$ so that $uf+vg=\text{gcd}(f,g)$.
\end{thm}
\begin{thm} [Chinese Remainder Theorem] If polynomials $Q_1,\ldots, Q_n\in K[x]$ are relatively prime, then the system $P\equiv R_i\pmod{Q_i}, 1\leq i\leq n$ has a unique solution modulo $Q_1\cdots Q_n$.
\end{thm}

\begin{enumerate}
\item
\end{enumerate}

\section{Algebraic Numbers}

Let $R$ be an integral domain and $K$ its fraction field, and $L$ a field containing $K$. 
A number $a$ in $L$ is said to be \emph{algebraic} over $K$ if it satisfies a nontrivial polynomial equation with coefficients in $K$. The number is an \emph{algebraic integer} if this polynomial can be chosen to be monic with coefficients in $R$. Unless otherwise specified, we work over $\mathbb{Z}$ and $\mathbb{Q}$.

\begin{thm}[Fundamental Theorem of Symmetric Polynomials]
Let $R$ be a ring (say, $\mathbb Z$), and $f(x_1,\ldots, x_n)$ a polynomial symmetric in all its variables. Then there exists a unique polynomial $g$ such that
\[f(x_1,\ldots, x_n)=g(s_1,\ldots, s_n)\]
where $s_j=\sum_{1\leq i_1<\ldots<i_j\leq n}x_{i_1}\cdots x_{i_j}$ are the elementary symmetric polynomials.
\end{thm}
\begin{proof}
Induct on the degree of $f$ and the number of variables.
\end{proof}

\begin{thm}
The \emph{minimal (irreducible) polynomial} $p$ of $a\in L$ is the monic polynomial of minimal degree in $K[x]$ that has $a$ as a root. Any polynomial in $K[x]$ that has $a$ as a root is a multiple of $p$.
\end{thm}
\begin{proof}
The polynomials in $K[x]$ that have $a$ as a root form an ideal. $K[x]$ is a principal ideal domain, so is generated by one element.
\end{proof}
The degree of the minimal polynomial of $a$ over $K$ is also called as the degree of $a$ over $K$. This is the dimension of $K(a)$ as a vector space over $K$; i.e. it takes $d$ elements $1,a,\ldots, a^{d-1}$ to generate $K(a)$ over $K$. The zeros of the minimal polynomial of $a$ are called the \emph{conjugates} of $a$.

\begin{thm}
The numbers in $L$ that are algebraic over $K$ form a field. The algebraic integers over $R$ form a ring.
\end{thm}
\begin{proof}
We need to show that if $a,b$ are algebraic numbers and $k\in K$ then so are $ka$, $a+b$, $ab$, and $1/a$. 

\noindent \textbf{Proof 1} Let $p,q$ be the minimal polynomials of $a,b$, let $a_1,\ldots, a_k$ be the conjugates of $a$ and $b_1,\ldots, b_l$ be the conjugates of $b$. Then the coefficients of
\[\prod_i (x-ka_i),\prod_{i,j} (x-(a_i+b_j)),\prod_{i,j} (x-(a_ib_j)),\prod_{i} (x-(1/a_i))\]
are symmetric in the $a_i$ and symmetric in the $b_j$ so by the Fundamental Theorem can be written in terms of the elementary symmetric polynomials in the $a_i$ and in the $b_j$. But by Vieta's Theorem these are expressible in terms of the coefficients of $p,q$, which are in $K$. Hence these polynomials have coefficients in $K$ and have $ka, a+b,ab,1/a$ as roots, as desired. If $a,b$ are algebraic integers so are $ra,r\in R$, $a+b$, $ab$ by noting that the coefficients of the first three polynomials are in $R$ and the polynomials are monic.

\noindent\textbf{Proof 2} Consider the field generated by $a,b$ over $K$. It is spanned by $a^ib^j$ for $0\leq i< k$ and $0\leq j< l$, and hence is finite-dimensional as a vector space. Hence for any $c$ in this field, $1,c,c^2,\ldots$ must satisfy a linear dependency relation, i.e. $c$ is algebraic. (The proof for algebraic integers is similar but more involved.)
\end{proof}
Note that the first proof gives us the additional fact that the conjugates of $a+b$ are among the $a_i+b_j$ and the conjugates of $ab$ are among the $a_ib_j$.

\begin{thm}[Rational Roots Theorem]
The possible rational roots of $a_nx^n+\cdots +a_0\in\mathbb Z[x]$ %where $a_0\neq 0$ 
are $\frac{p}{q}$ where $p|a_0$ and $q|a_n$. Thus all algebraic integers that are rational are also in $\mathbb{Z}$ (which we will call rational integers).
\end{thm}

\subsection{Problems}

\begin{enumerate}
\item Suppose that $f\in\mathbb{Z}[x]$ is irreducible and has a root of absolute value at least $\frac 32$. Prove that if $\alpha$ is a root of $f$ then $f(\alpha^3+1)\neq 0$.

\item Let $a_1,\ldots, a_n$ be algebraic integers with degrees $d_1,\ldots, d_n$. Let $a_1',\ldots, a_n'$ be the conjugates of $a_1,\ldots, a_n$ with greatest absolute value. Let $c_1,\ldots, c_n$ be integers. Prove that if the LHS of the following expression is not zero, then
\[|c_1a_1+\ldots+c_na_n|\geq\left(\frac{1}{|c_1a_1'|+\cdots+|c_na_n'|}\right)^{d_1d_2\cdots d_n-1}.\]
For example, 
\[|c_1+c_2\sqrt{2}+c_3\sqrt{3}|\geq \left(\frac{1}{|c_1|+|2c_2|+|2c_3|}\right)^{3}.\]

\item Let $p$ be a prime and consider $k$ $p$th roots of unity whose sum is not 0. Prove that the absolute value of their sum is at least $\frac{1}{k^{p-2}}$.
\end{enumerate}

\section{Cyclotomic and Chebyshev Polynomials}

The $n$th cyclotomic polynomial is defined by
\[ \Phi_n(x)=\prod_{0\leq j <n,\text{gcd}(j,n)=1}(x-e^{\frac{2\pi ij}{n}})\]
Equivalently, it can be defined by the recurrence $\Phi_0(x)=1$ and
\[ \Phi_n(x)=\frac{x^n-1}{\prod_{m\mid n,m<n}\Phi_m(x)}.\]
Hence, it has integer coefficients.
\begin{thm}
The cyclotomic polynomials are irreducible over $\mathbb{Q}[x]$.
\end{thm}
\begin{proof}
We need the following lemma:

Suppose $\omega$ is a primitive $n$th root of unity, and that its minimal polynomial is $g(x)$. Let $p$ be a prime not dividing $n$. Then $\omega^p$ is a root of $g(x)=0$.

Since $\Phi_n(\omega)=0$, we can write $\Phi_n=fg$. If $g(\omega^p)\neq 0$ then $f(\omega^p)=0$. Since $\omega$ is a zero of $f(x^p)$, $f(x^p)$ factors as
\[f(x^p)=g(x)h(x)\]
for some polynomial $h\in \mathbb{Z}[x]$.

Now, in $\mathbb{Z}/p\mathbb{Z}[x]$ note $(a_1+\ldots+a_k)^p=a_1^p+\ldots +a_k^p$ ($\Phi:a\to a^p$ is a homomorphism in $\mathbb{Z}/p\mathbb{Z}[x]$ since $(P+Q)^p=P^p+Q^p$ by the binomial theorem.).
%expand using the multinomial theorem and note $p\mid\binom{p}{i}$ for $0<i<p$
Hence
\[g(x)h(x)\equiv f(x^p)\equiv f(x)^p \pmod{p}.\]
%Since $\omega $ is a root of $f(x^p)$, $\omega$ is also a root of $f(x)
Hence $f(x)$ and $g(x)$ share a factor modulo $p$.
However, the derivative of $x^n-1$ modulo $p$ is $nx^{n-1}\neq 0$, showing that $x^n-1$ has no repeated irreducible factor modulo $p$; hence $\Phi_n$ has no repeated factor modulo $p$. Since $\Phi_n=fg$, this produces a contradiction.

Therefore $g(\omega^p)=0$, as needed.

Any primitive $n$th root is in the form $\omega^{k}$ for $k$ relatively prime to $n$. Writing the prime factorization of $k$ as $p_1\cdots p_m$, we get by the lemma that $\omega^{p_1},\omega^{p_1p_2},\ldots, \omega^{p_1\cdots p_m}$ are all roots of $g$. Hence $g$ contains all primitive $n$th roots of unity as roots, and $\Phi_n=g$ is irreducible.
\end{proof}
Application: Special case of Dirichlet's Theorem: Given $n$ there are infinitely many primes $p\equiv 1\pmod{n}$.

The Chebyshev polynomials are defined by the recurrence $T_0(x)=1, T_1(x)=x, T_{i+1}(x)=2xT_i(x)-T_{i-1}(x)$ for $i\geq 1$. They satisfy
\[T_n(\cos\theta)=\cos n\theta\]
since $\cos((n+1)\theta)=2\cos\theta \cos n\theta-\cos(n-1)\theta$.
Furthermore,
\[T_n\left(\frac{1}{2}\left(x+\frac{1}{x}\right)\right)=\frac{1}{2}\left(x^n+\frac{1}{x^n}\right).\]

The roots of $T_n(x)$ are $\cos \left(\frac{\pi}{n}+\frac{2\pi k}{n}\right), 0\leq k<n$.

\noindent \textbf{Problems E}
\begin{enumerate}
\item Let $p$ be a prime. Prove that any equiangular $p$-gon with rational side lengths is regular.

\item Suppose $P$ is polynomial of degree at most 7 so that
\begin{align*}
P\pa{\frac{\sqrt 2+\sqrt 6}{4}}&=-\frac{\sqrt 6-\sqrt 2}{4}\\
P\pa{\frac{\sqrt 3}{2}}&=-\frac{\sqrt{3}}{2}\\
P\pa{\rc{2}}&=\frac{1}{2}\\
P\pa{\frac{\sqrt 6-\sqrt 2}{4}}&=-\frac{\sqrt 2+\sqrt 6}{4}\\
P\pa{-\frac{\sqrt 2+\sqrt 6}{4}}&=\frac{\sqrt 6-\sqrt 2}{4}\\
P\pa{-\frac{\sqrt 3}{2}}&=\frac{\sqrt{3}}{2}\\
P\pa{-\rc{2}}&=-\frac{1}{2}\\
P\pa{-\frac{\sqrt 6-\sqrt 2}{4}}&=\frac{\sqrt 2+\sqrt 6}{4}
\end{align*}
Find $P(5/4)$.

\item (IMO) The sequence of polynomials $f_n(x)$ is defined as follows:
\[f_0(x) = x\textrm{ and }f_n(x) = f_{n-1}(x)^2-2.\]
Show that for all positive integers $n$, the equation $f_n(x)=x$ has
all real distinct roots.

\item (Komal) Prove that there exists a positive integer $n$ so that any prime divisor of $2^n-1$ is smaller that $2^{\frac{n}{1993}}-1$.

\item Find all rational $p\in [0,1]$ such that $\cos p\pi$ is...
\begin{enumerate}
\item rational
\item the root of a quadratic polynomial with rational coefficients
\end{enumerate}

\item (China) Prove that there are no solutions to $2\cos p\pi=\sqrt{n+1}-\sqrt{n}$ for rational $p$ rational and positive integer $n$.

\item (TST 2007/3) Let $\theta$ be an angle in the interval $(0,\pi/2)$. Given that $\cos \theta$ is irrational and that $\cos k\theta$ and $\cos[(k+1)\theta]$ are both rational for some positive integer $k$, show that $\theta=\pi/6$.

\item (Chebyshev) Let $p(x)$ be a real polynomial of degree $n\geq 1$ with leading coefficient 1. Then \[\max_{-1\leq x\leq 1} |p(x)|\geq \frac{1}{2^{n-1}}.\]


\item Prove that $\cos\frac{\pi}{4n}\cdot \cos\frac{3\pi}{4n}\cdots \cos\frac{(2n-1)\pi}{4n}=\frac{1}{2^{n-\frac 12}}$.

\end{enumerate}

\noindent \textbf{(F) Polynomials in Number Theory}
\begin{itemize}
\item (Lagrange) A polynomial of degree $n$ over a field can have at most $n$ zeros.
\item To evaluate a sum or product it may be helpful to find a polynomial with those terms as zeros and use Vieta's relations.
\begin{thm}
Let $r_1,\ldots, r_n$ be the roots of $\sum_{i=0}^n a_ix^i$, and let \[s_j=\sum_{1\leq i_1<\ldots<i_j\leq n}r_{i_1}\cdots r_{i_j}.\] Then $s_j=(-1)^j\frac{a_{n-j}}{a_n}$.
\end{thm}
\end{itemize}
\begin{enumerate}
\item (Wolstenholme) Prove that $\binom{pa}{pb}\equiv \binom{a}{b}\pmod{p^3}$ for prime $p\geq 5$.

\item Prove that for prime $p\geq 5$,
\[p^2|(p-1)!\left(1+\frac{1}{2}+\cdots +\frac{1}{p-1}\right).\]

\item (APMO 2006/3) Prove that for prime $p\geq 5$, $\binom{p^2}{p}\equiv p \pmod{p^5}$.

\item (ISL 2005/N3) Let $a,b,c,d,e,f$ be positive integers. Suppose that the sum $S=a+b+c+d+e+f$ divides both $abc+def$ and $ab+bc+ca-de-ef-fd$. Prove that $S$ is composite.

\item (China TST 2009/3) Prove that for any odd prime $p$, the number of positive integers $n$ satisfying $p\mid n!+1$ is less than or equal to $cp^{\frac{2}{3}}$, where $c$ is a constant independent of $p$.

\item (TST 2002/2) Let $p$ be a prime number greater than 5. For any positive integer $x$, define
\[f_p(x)=\sum_{k=1}^{p-1}\frac{1}{(px+k)^2}.\]
Prove that for all positive integers $x$ and $y$ the numerator of $f_p(x)-f_p(y)$, when written in lowest terms, is divisible by $p^3$.
\end{enumerate}
\end{document}