\documentclass[12pt,a4paper,twoside]{article}
\usepackage{amsmath}
\usepackage{amssymb}
\usepackage{amsthm}
\usepackage{array}
\usepackage{amsfonts}
\usepackage{ctable,booktabs}
%\usepackage[pdftex]{graphicx}
\usepackage{enumerate}
\usepackage{fancyhdr}
\usepackage{float}
\usepackage{fullpage}
\usepackage[margin=1in]{geometry}
\usepackage{mathrsfs}%Math script
\usepackage{sidecap}
\usepackage{tabularx} 
\usepackage{verbatim}
\usepackage{wrapfig}
\usepackage[all,cmtip]{xy}%Commutative diagrams
	\headheight 0cm
	\setlength{\headsep}{18pt}
	\setlength{\headheight}{15.2pt}

\newtheoremstyle{norm}
{3pt}
{3pt}
{}
{}
{\bf}
{:}
{.5em}
{}

\theoremstyle{norm}
\newtheorem{thm}{Theorem}[section]
\newtheorem{lem}[thm]{Lemma}
\newtheorem{df}[thm]{Definition}
\newtheorem{rem}[thm]{Remark}
\newtheorem{st}{Step}
\newtheorem{pr}[thm]{Proposition}
\newtheorem{cor}[thm]{Corollary}
\newtheorem{conj}[thm]{Conjecture}
\newtheorem{clm}[thm]{Claim}
\newtheorem{exr}[thm]{Exercise}
\newtheorem{ex}[thm]{Example}
\newtheorem{prb}[thm]{Problem}

%Math blackboard, fraktur, and script commonly used letters
\newcommand{\A}[0]{\mathbb{A}}
\newcommand{\C}[0]{\mathbb{C}}
\newcommand{\sC}[0]{\mathcal{C}}
\newcommand{\cE}[0]{\mathscr{E}}
\newcommand{\F}[0]{\mathbb{F}}
\newcommand{\cF}[0]{\mathscr{F}}
\newcommand{\sF}[0]{\mathscr{F}}
\newcommand{\cG}[0]{\mathscr{G}}
\newcommand{\sH}[0]{\mathscr H}
\newcommand{\Hq}[0]{\mathbb{H}}
\newcommand{\N}[0]{\mathbb{N}}
\newcommand{\Pj}[0]{\mathbb{P}}
\newcommand{\sO}[0]{\mathcal{O}}
\newcommand{\cO}[0]{\mathscr{O}}
\newcommand{\Q}[0]{\mathbb{Q}}
\newcommand{\R}[0]{\mathbb{R}}
\newcommand{\Z}[0]{\mathbb{Z}}
%Lowercase
\newcommand{\ma}[0]{\mathfrak{a}}%ideal a
\newcommand{\mb}[0]{\mathfrak{b}}
\newcommand{\fg}[0]{\mathfrak{g}}
\newcommand{\vi}[0]{\mathbf{i}}%vector i
\newcommand{\vj}[0]{\mathbf{j}}
\newcommand{\vk}[0]{\mathbf{k}}
\newcommand{\mm}[0]{\mathfrak{m}}%ideal m
\newcommand{\mfp}[0]{\mathfrak{p}}
\newcommand{\mq}[0]{\mathfrak{q}}
\newcommand{\mr}[0]{\mathfrak{r}}
%More sequences of letters
\newcommand{\fq}[0]{\mathbb{F}_q}
\newcommand{\fqt}[0]{\mathbb{F}_q^{\times}}
\newcommand{\sll}[0]{\mathfrak{sl}}
%Shortcuts for symbols
\newcommand{\nin}[0]{\not\in}
\newcommand{\opl}[0]{\oplus}
\newcommand{\ot}[0]{\otimes}
\newcommand{\rc}[1]{\frac{1}{#1}}
\newcommand{\sub}[0]{\subset}
\newcommand{\subeq}[0]{\subseteq}
\newcommand{\supeq}[0]{\supseteq}
\newcommand{\nsubeq}[0]{\not\subseteq}
\newcommand{\nsupeq}[0]{\not\supseteq}
%Arrows
\newcommand{\lar}[0]{\leftarrow}
\newcommand{\ra}[0]{\rightarrow}
\newcommand{\rra}[0]{\rightrightarrow}
\newcommand{\hra}[0]{\hookrightarrow}
\newcommand{\send}[0]{\mapsto}
%Shortcuts for greek letters
\newcommand{\al}[0]{\alpha}
\newcommand{\be}[0]{\beta}
\newcommand{\ga}[0]{\gamma}
\newcommand{\Ga}[0]{\Gamma}
\newcommand{\de}[0]{\delta}
\newcommand{\ep}[0]{\varepsilon}
\newcommand{\eph}[0]{\frac{\varepsilon}{2}}
\newcommand{\ept}[0]{\frac{\varepsilon}{3}}
\newcommand{\la}[0]{\lambda}
\newcommand{\La}[0]{\Lambda}
\newcommand{\ph}[0]{\varphi}
\newcommand{\rh}[0]{\rho}
\newcommand{\te}[0]{\theta}
\newcommand{\om}[0]{\omega}
\newcommand{\Om}[0]{\Omega}
%Brackets
\newcommand{\ab}[1]{\left| {#1} \right|}
\newcommand{\an}[1]{\langle {#1}\rangle}
\newcommand{\ba}[1]{\left[ {#1} \right]}
\newcommand{\bc}[1]{\left\{ {#1} \right\}}
\newcommand{\ce}[1]{\left\lceil {#1}\right\rceil}
\newcommand{\fl}[1]{\left\lfloor {#1}\right\rfloor}
\newcommand{\pa}[1]{\left( {#1} \right)}
%Text
\newcommand{\btih}[1]{\text{ by the induction hypothesis{#1}}}
\newcommand{\bwoc}[0]{by way of contradiction}
\newcommand{\by}[1]{\text{by~(\ref{#1})}}
\newcommand{\ore}[0]{\text{ or }}
\newcommand{\wog}[0]{ without loss of generality }
\newcommand{\Wog}[0]{ Without loss of generality }
%Functions, etc.
\newcommand{\Ann}{\operatorname{Ann}}
\newcommand{\AP}{\operatorname{AP}}
\newcommand{\Ass}{\operatorname{Ass}}
\newcommand{\chr}{\operatorname{char}}
\newcommand{\cis}{\operatorname{cis}}
\newcommand{\Cl}{\operatorname{Cl}}
\newcommand{\Der}{\operatorname{Der}}
\newcommand{\End}{\operatorname{End}}
\newcommand{\Ext}{\operatorname{Ext}}
\newcommand{\Frac}{\operatorname{Frac}}
\newcommand{\FS}{\operatorname{FS}}
\newcommand{\GL}{\operatorname{GL}}
\newcommand{\Hom}{\operatorname{Hom}}
\newcommand{\chom}[0]{\mathscr{H}om}
\newcommand{\Ind}[0]{\text{Ind}}
\newcommand{\im}[0]{\text{im}}
\newcommand{\nil}[0]{\operatorname{nil}}
\newcommand{\Proj}{\operatorname{Proj}}
\newcommand{\Rad}{\operatorname{Rad}}
\newcommand{\Res}[0]{\text{Res}}
\newcommand{\sign}{\operatorname{sign}}
\newcommand{\SL}{\operatorname{SL}}
\newcommand{\Spec}{\operatorname{Spec}}
\newcommand{\Specf}[2]{\Spec\pa{\frac{k[{#1}]}{#2}}}
\newcommand{\spp}{\operatorname{sp}}
\newcommand{\spn}{\operatorname{span}}
\newcommand{\Supp}{\operatorname{Supp}}
\newcommand{\Tor}{\operatorname{Tor}}
\newcommand{\tr}[0]{\text{Tr}}
%Commutative diagram shortcuts
\newcommand{\commsq}[8]{\xymatrix{#1\ar[r]^{#6}\ar[d]^{#5} &#2\ar[d]^{#7} \\ #3 \ar[r]^{#8} & #4}}
%Makes a diagram like this
%1->2
%|    |
%3->4
%Arguments 5, 6, 7, 8 on arrows
%  6
%5  7
%  8
\newcommand{\pull}[9]{
#1\ar@/_/[ddr]_{#2} \ar@{.>}[rd]^{#3} \ar@/^/[rrd]^{#4} & &\\
& #5\ar[r]^{#6}\ar[d]^{#8} &#7\ar[d]^{#9} \\}
\newcommand{\back}[3]{& #1 \ar[r]^{#2} & #3}
%Syntax:\pull 123456789 \back ABC
%1=upper left-hand corner
%2,3,4=arrows from upper LH corner, going down, diagonal, right
%5,6,7=top row (6 on arrow)
%8,9=middle rows (on arrows)
%A,B,C=bottom row
%Other
\newcommand{\op}{^{\text{op}}}
\newcommand{\fp}[1]{^{\underline{#1}}}%Falling power
\newcommand{\rp}[1]{^{\overline{#1}}}
\newcommand{\rd}[0]{_{\text{red}}}
\newcommand{\pre}[0]{^{\text{pre}}}
\newcommand{\pf}[2]{\pa{\frac{#1}{#2}}}%Shortcut for fraction with parentheses
\newcommand{\pd}[2]{\frac{\partial #1}{\partial #2}}%Partial derivatives
%Matrices
\newcommand{\coltwo}[2]{
\left[
\begin{array} {c}
{#1}\\
{#2} 
\end{array}
\right]}
\newcommand{\matt}[4]{
\begin{pmatrix} {cc}
{#1}&{#2}\\
{#3}&{#4}
\end{pmatrix}
}
\newcommand{\smatt}[4]{
\left[\begin{smallmatrix} {cc}
{#1}&{#2}\\
{#3}&{#4}
\end{smallmatrix}\right]
}
\newcommand{\colthree}[3]{
\begin{pmatrix} {c}
{#1}\\
{#2}\\
{#3}
\end{pmatrix}
}
%Page breaks in equations
\allowdisplaybreaks[1]

\rhead{\qquad}
%\lhead{}\rhead{}\cfoot{}\chead{} 
%\lfoot[\fancyplain{}{\bfseries\thepage}]{}
%\rfoot[]{\thepage}
%\lfoot[\thepage]{}

%\setlength{\oddsidemargin}{0.55in}
%\setlength{\evensidemargin}{0.55in}


%\fi
\pagestyle{fancy}

%OMC <year>
\lhead{\textit{OMC 2011}}

% LECTURE TITLE
\chead{Polynomials}

%%%%%%%%%%
% LECTURE NUMBER
%%%%%%%%%%
\rhead{\textit{Lecture 8}}


%%%%%%%%%%
% CONTENT
%
% Here is where you place the problems and any other content. If you are making a problem set, use the \item command to create a new problem. 
%
%%%%%%%%%%
\begin{document}
\title{Solutions to Lecture $8$ --- Polynomials}% !! Remember to change the lecture number
\author{Holden Lee}
\date{2/4/2011}% !! Remember to change the date
\maketitle
\thispagestyle{empty}
\section{Values and Zeros}
\begin{enumerate}
\item This time it's easier to guess the solution. The polynomial 
\[Q(x)=\sum_{i=0}^n \binom{x}{i} (r-1)^i\]%\binom{x}{0}+\cdots+ \binom{x}{n}$ 
has degree $n$ and by the Binomial Theorem, satisfies the given conditions. Since $P(x)=Q(x)$ for $n+1$ values of $x$, actually $P,Q$ are the same polynomial, and 
\[P(n+1)=Q(n+1)=\left(\sum_{i=0}^{n+1} \binom{n+1}{i} (r-1)^i\right)-(r-1)^{n+1}=r^{n+1}-(r-1)^{n+1}.\]

\item Letting ray $OQ_1$ be the positive real axis, $Q_i$ represent the $n$th roots of unity $\omega^i$ in the complex plane. Hence $PQ_i$ equals $|2-\omega^i|$. The roots of $x^n-1=0$ are just the $n$th roots of unity, so $x^n-1=\prod_{i=0}^{n-1} (x-\omega^i)$. Plugging in $x=2$ gives $\prod_{k=1}^n |PQ_i|=2^n-1$.

\item The given condition says
\begin{equation}f(x)^2=x(x-1)\cdots (x-n)Q(x)+x^2+1
\label{a1-1}
\end{equation}
for some polynomial $f(x)$ of degree at most $n$. Plugging $x=0,1,\ldots ,n$ into~(\ref{a1-1}) gives
\begin{equation}f(x)=\pm\sqrt{x^2+1},\text{ when }x=0,1,\ldots, n.
\label{a1-2}
\end{equation}
The following is key:
Given $n+1$ points $(x_0,y_0),\ldots, (x_n,y_n)$ with distinct $x$-coordinates, there exists exactly one polynomial $f$ of degree at most $n$ so that $f(x_i)=y_i$ for $i=0,1,\ldots, n$.

Applying this to~(\ref{a1-2}) we get $2^{n+1}$ possibilities for $f(x)$ since we have 2 choices of sign for each of $x=0,1,\ldots, n$. If $f(x)$ is a solution to~(\ref{a1-2}) then so is $-f(x)$; we get $2^n$ possibilities for $f(x)^2$. Solve~(\ref{a1-1}) to get $2^n$ possibilities for $Q(x)$:
\[Q(x)=\frac{f(x)^2-x^2-1}{x(x-1)\cdots (x-n)}\]
Each such polynomial is a valid solution because $f(x)^2-x^2-1$ is zero at $x=0,1,\ldots, n$ and hence is divisible by $x(x-1)\cdots (x-n)$.

\item Clearing denominators,
\[\sum_{i=1}^5 \left[a_i x\prod_{i\neq j, 1\leq j\leq 5} (x+j)\right]-(x+1)(x+2)(x+3)(x+4)(x+5)=0\]
for $x=1,4,9,16,25$. Let $f(x)$ denote the LHS. Since $f(x)=0$ has the roots $x=1,4,9,16,25$, we conclude that $(x-1)(x-4)\cdots (x-25)$ divides $f(x)$. Since $f(x)$ has degree at most 5, \[f(x)=k(x-1)(x-4)(x-9)(x-16)(x-25)\] for some constant $k$. However, equating
\[f(0)=[-(x+1)(x+2)(x+3)(x+4)(x+5)]_{x=0}=-5!\]
and $f(0)=-k\cdot 5!^2$ gives $k=\frac{1}{5!}$. Thus
\[f(x)=\frac{1}{5!}(x-1)(x-4)(x-9)(x-16)(x-25).\]
Then 
\begin{align*}
\sum_{i=1}^5 \frac{a_i}{6^2+i}&=\frac{f(6^2)}{x(x+1)(x+2)(x+3)(x+4)(x+5)|_{x=6^2}}-\frac{1}{6^2}\\
&=\frac{187465}{6744582}.
\end{align*}

\item 
Given two points $(x_1,y_1)$ and $(x_2,y_2)$ the equation 
\[\frac{y-y_1}{y_2-y_1}=\frac{x-x_1}{x_2-x_1} \implies
(y-y_1)(x_2-x_1)=(x-x_1)(y_2-y_1)\] represents the line passing through these two points (it is a linear equation satisfied by the coordinares of the two points). It follows that three points $(x_1,y_1)$, $(x_2,y_2)$ and $(x_3,y_3)$ lie on the same line if and only if the condition 
\begin{equation}(y_3-y_1)(x_2-x_1)= (x_3-x_1)(y_2-y_1) 
\label{um02-3}
\end{equation}
holds. Now suppose that $(x_1,y_1)$, $(x_2,y_2)$ and $(x_3,y_3)$ represent the (changing) coordinates of the three ducks as they waddle along their paths. Each coordinate is a linear function of time $t$, so~(\ref{um02-3}) is an equation in $t$ of degree at most 2 (i.e., is a quadratic). If such an equation has more than 2 solutions then it must reduce to an identity and thus hold true for all values of $t$. That is, if the ducks are in a row at more than two times, then they are always in a row. 

\item Note that $g(x)$ is one of the 16 integer divisors of 2008 for each of the 81 integer roots. There must be at least 6 roots of $f(x)$ for which $g(x)$ has the same value. Since $g(x)$ is nonconstant, its degree must be greater than 5.

\item (Official solution) Let $p(x)$ be the monic real polynomial of degree $n$. If $n=1$, then $p(r)=r+a$ for some real number $a$, and $p(x)$ is the average of $x$ and $x+2a$, each of which has 1 real root. Now we assume that $n>1$. Let
\[g(x)=(x-2)(x-4)\cdots (x-2(n-1)).\]
The degree of $g(x)$ is $n-1$. Consider the polynomials
\[q(x)=x^n-kg(x), r(x)=2p(x)-q(x)=2p(x)-x^n+kg(x).\]
We will show that for large enough $k$ these two polynomials have $n$ real roots. Since they are monic and their average is clearly $p(x)$, this will solve the problem.

Consider the values of the polynomial $g(x)$ at $n$ points $x=1,3,5,\ldots, 2n-1$. These values alternate in sign and are at least 1 (since at most two of the factors have magnitude 1 and the others have magnitude at least 2). On the other hand, there is a constant $c>0$ such that for $0\leq x\leq n$, we have $|x^n|<c$ and $|2p(x)-x^n|<c$. Take $k>c$. Then we see that $q(x)$ and $r(x)$ evaluated at $n$ points $x=1,3,5,\ldots, 2n-1$ alternate in sign. Thus $q(x)$ and $r(x)$ each has at least $n-1$ real roots. However since they are polynomials of degree $n$, they must have $n$ real roots, as desired.

\item Without loss of generality, suppose $\deg(f)<\deg(g)$. Let $r_1,\ldots, r_k$ be the distinct roots of $f$ and let $s_1,\ldots, s_l$ be the distinct roots of $f-1$.

We claim that $k+l\geq \deg(n)+1$. Indeed, suppose
\[f(x)=(x-r_1)^{p_1}\cdots (x-r_k)^{p_k}.\]
Then
\[(x-r_1)^{p_1-1}\cdots (x-r_k)^{p_k-1}\mid f'.\]
Similarly, if 
\[f(x)-1=(x-s_1)^{q_1}\cdots (x-s_l)^{q_l},\]
then
\[(x-s_1)^{q_1-1}\cdots (x-s_l)^{q_l-1}\mid f'.\]
Since the roots of $f$ and $f-1$ are distinct,
\[(x-r_1)^{p_1-1}\cdots (x-r_k)^{p_k-1}(x-s_1)^{q_1-1}\cdots (x-s_l)^{q_l-1}\mid f'.\]
Since $f'$ has degree $n-1$, 
\[(p_1-1)+\ldots +(p_k-1)+(q_1-1)+\ldots (q_l-1)\leq n-1.\]
Since $p_1+\ldots +p_k=q_1+\ldots +q_l=n$, this gives $(n-k)+(n-l)\leq n-1$, or $k+l\geq n+1$.

Now $f-g$ has degree at most $n$ and has at least $n+1$ distinct roots $r_1,r_2,\ldots, r_k,s_1,\ldots, s_l$, so it must be identically 0, and $f=g$.
\end{enumerate}

\section{Symmetric Polynomials and Vieta's Formulas}
\begin{enumerate}
\item Note the polynomial has degree 2000 since the $x^{2001}$ terms cancel out. By the Binomial Theorem, the coefficients of $x^{2000}$ and $x^{1999}$ are $2001\pf12$ and $-\binom{2001}{2}\pf12^2$, respectively. By Vieta's formula the sum of the roots is
\[
-\frac{-\binom{2001}{2}\pf12^2}{2001\pf12}=500.
\]
\item Using Vieta's formulas with the roots $r_i$,
\begin{align*}
\left(\sum\frac{1}{r_1^2}\right)&=\left(\sum\frac{1}{r_1}\right)^2-2\left(\sum\frac{1}{r_1r_2}\right)\\
&=\left(\frac{\sum r_1r_2r_3r_4}{r_1r_2r_3r_4r_5}\right)^2
-2\left(\frac{\sum r_1r_2r_3}{r_1r_2r_3r_4r_5}\right)\\
&=\frac{9^2}{(-11)^2}-\frac{2(-7)}{-11}=-\frac{73}{121}
\end{align*}
\item
\begin{enumerate}
\item The roots $r_1,r_2,r_3$ satisfy the equation $\pf1{x}^3+a\pf1{x}^2+b\pf1{x}+c=0$. Clearing denominators, they are roots to $cx^3+bx^2+ax+1$, and hence to \[x^3+\frac{b}{c}x^2+\frac{a}{c}x+\frac{1}{c}.\]
\item By Vieta's formula, $r_1+r_2+r_3=-a$, $r_1r_2+r_2r_3+r_3r_1=b$, and $r_1r_2r_3=-c$. We calculate the elementary symmetric sums in $r_1+r_2,r_2+r_3,r_3+r_1$:
\begin{align*}
(r_1+r_2)+(r_2+r_3)+(r_3+r_1)&=-2a\\
(r_1+r_2)(r_2+r_3)+(r_2+r_3)(r_3+r_1)\\+(r_3+r_1)(r_1+r_2)
&=(r_1+r_2+r_3)^2+(r_1r_2+r_2r_3+r_3r_1)\\
&=a^2+b\\
(r_1+r_2)(r_2+r_3)(r_3+r_1)&=(r_1+r_2+r_3)(r_1r_2+r_2r_3+r_3r_1)-r_1r_2r_3\\
&=-ab+c
\end{align*}
Hence by Vieta's formulas (using the roots to get the coefficients), $r_1+r_2,r_2+r_3,r_3+r_1$ are roots of
\[
x^3+2ax^2+(a^2+b)x+(ab-c).
\]
\item We calculate the elementary symmetric sums in $r_1^2,r_2^2,r_3^2$:
\begin{align*}
r_1^2+r_2^2+r_3^2&=(r_1+r_2+r_3)^2-2(r_1r_2+r_2r_3+r_3r_1)=a^2-2b\\
r_1^2r_2^2+r_2^2r_3^2+r_3^2r_1^2&=(r_1r_2+r_2r_3+r_3r_1)^2 -2r_1r_2r_3 (r_1+r_2+r_3)=b^2-2ac\\
r_1^2r_2^2r_3^2&=c^2
\end{align*}
Hence by Vieta's formulas, $r_1^2,r_2^2,r_3^2$ are roots of
\[
x^3+(2b-a^2)x^2+(b^2-ac)x-c^2.
\]
\end{enumerate}
\item The coefficient of $x^2$ is 0 so $r+s+t=0$. Using $rst=\frac{-2008}{8}$, we get
\begin{align*}(r+s)^3+(s+t)^3+(t+r)^3&=2(r+s+t)^3-3(r+s+t)(rs+st+tr) -3rst\\
&=-3\pf{-2008}{8}=-753.\end{align*}
\item Let $y=2^{111x}$; the equation becomes $\frac{1}{4}y^3+4y=2y^2+1$ which rearranges to $y^3-8y^2+16y-4=0$. Let $y_1,y_2,y_3$ be the roots of this equation and $x_1,x_2,x_3$ be the solutions to the original equation. Then
\[2^{111(x_1+x_2+x_3)}=y_1y_2y_3=4\]
by Vieta's formula so $x_1+x_2+x_3=\frac1{111}\log_2 4=\frac 2{111}$ and the answer is 113.
\item To simplify the calculation, we first divide $P(x)$ by $Q(x)$ to obtain
\[P(x)=Q(x)(x^2+1)+x^2-x+1.\]
Thus
\[\sum_{i=1}^4 P(z_i)=\sum_{i=1}^4 Q(z_i)(z_i^2+1)+\sum_{i=1}^4 (z_i^2-z_i+4)=\sum_{i=1}^4 (z_i^2-z_i+4).\]
The first and second elementary symmetric sums equal 1 and $-1$ by Vieta. Hence the above sum equals
\[
\pa{\sum_{i=1}^4 z_i}^2-2\sum_{1\leq i<j\leq 4} z_iz_j-\sum_{i=1}^4 z_i+4=1+2-1+4=6.
\]
\end{enumerate}
\section{Fundamental Theorem of Algebra}
\begin{enumerate}
\item Write the equation as 
\[
(g(x)+ih(x))(g(x)-ih(x))=\frac{x^{20}-1}{x^2-1}.
\]
Counting the number of possibilities for $(f(x),g(x))$ is the same as counting the number of possibilities for $f(x)=g(x)+ih(x)$. 
Thus we need to count the number of complex polynomials $f(x)$ such that 
\[
f(x)\bar{f}(x)=\frac{x^{20}-1}{x^2-1}.
\]
The zeros of $\frac{x^{20}-1}{x^2-1}$ can be split in complex conjugate pairs $P_1,\ldots, P_9$, since they are the nonreal 20th roots of unity. If $r$ is a zero of $f(x)$ then $\bar r$ is a zero of $\bar{f}(x)$. Thus $f(x)$ must have as a zero one number in each pair $P_i$, and $\bar{f}(x)$ has as its zeros the other number in each pair $P_i$. There are $2^9=512$ choices for which zeros in each pair to choose as zeros of $f(x)$. The answer is 512.
\end{enumerate}
\end{document}